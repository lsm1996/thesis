% Chapter Template

\chapter{Insecure Computing} % Main chapter title

\label{Chapter1} % Change X to a consecutive number; for referencing this chapter elsewhere, use \ref{ChapterX}

%----------------------------------------------------------------------------------------
%	SECTION 1
%----------------------------------------------------------------------------------------

\section{Insecure Computing: Design of a Secure Smartphone PCB}

The last few years has seen a disturbing upward trend in the number and severity of security vulnerabilities in modern computing devices. Whereas traditionally these security vulnerabilities have existed primarily in software, the ever-growing complexity of computer hardware, and in particular CPU designs, now means that we are beginning to see many more hardware-based vulnerabilities.  Whereas software vulnerabilities can be quickly and economically patched, this is not always the case for hardware-based vulnerabilities, as it may require replacement of the defective hardware.  
The recent Spectre and Meltdown vulnerabilities are headline examples of this trend and the difficulties that such vulnerabilities create. Intel and the other venders are still struggling to produce patches that do not reduce the reliability or performance of their processors, while mitigating as much of the threat posed by these vulnerabilities as possible.  At the end of the day, however, the only fully effective solution is for them to develop a new generation of processors that have the defects corrected in the silicon – an extremely expensive and time-consuming process – and even then, it is likely that the performance of those processors will be impaired, precisely because it is performance optimisations that are the direct cause of the vulnerabilities.  Further, even if we were to assume that corrected hardware were produced, it too could be found to suffer from one or more hitherto unknown vulnerabilities, and it too require replacement.  Put another way, the window of vulnerability for a hardware defect is typically orders of magnitude longer than for a software vulnerability. When commerce, personal privacy and security as well as national security depends on the correct and secure operation of computers, we clearly have a problem of considerable importance before us.
Modern computers are designed to achieve increasing performance and functionality through the easiest route. This is typically done by adding in extra layers of code, consequently increasing sophistication and complicating the system. The processors of these modern computers are optimised using speculative execution, which is a technique used by high-speed processors in order to increase performance by guessing future likely execution paths (https://spectreattack.com/spectre.pdf). Unlike older computers, this adds another level of complexity to the computer. This complexity however can be detrimental to security, due to vulnerabilities in the hardware being able to be exploited. There have been a number of recent cases where vulnerabilities have been exploited, in specific the Spectre and Meltdown hardware attacks. 
Spectre attacks involve inducing a victim to speculatively perform operations that would not occur during correct program execution, which consequently leak the victim’s confidential information to the adversary via a side channel (https://spectreattack.com/spectre.pdf).  
Meltdown attacks also break the fundamental isolation between the user applications and the operating system. The attack allows a program to access the memory of the affected computer. Which in turn can give away secrets of other programs and the operating system (https://meltdownattack.com/meltdown.pdf). 


%-----------------------------------
%	SUBSECTION 1
%-----------------------------------
\subsection{Subsection 1}

Nunc posuere quam at lectus tristique eu ultrices augue venenatis. Vestibulum ante ipsum primis in faucibus orci luctus et ultrices posuere cubilia Curae; Aliquam erat volutpat. Vivamus sodales tortor eget quam adipiscing in vulputate ante ullamcorper. Sed eros ante, lacinia et sollicitudin et, aliquam sit amet augue. In hac habitasse platea dictumst.

%-----------------------------------
%	SUBSECTION 2
%-----------------------------------

\subsection{Subsection 2}
Morbi rutrum odio eget arcu adipiscing sodales. Aenean et purus a est pulvinar pellentesque. Cras in elit neque, quis varius elit. Phasellus fringilla, nibh eu tempus venenatis, dolor elit posuere quam, quis adipiscing urna leo nec orci. Sed nec nulla auctor odio aliquet consequat. Ut nec nulla in ante ullamcorper aliquam at sed dolor. Phasellus fermentum magna in augue gravida cursus. Cras sed pretium lorem. Pellentesque eget ornare odio. Proin accumsan, massa viverra cursus pharetra, ipsum nisi lobortis velit, a malesuada dolor lorem eu neque.

%----------------------------------------------------------------------------------------
%	SECTION 2
%----------------------------------------------------------------------------------------

\section{Main Section 2}

Sed ullamcorper quam eu nisl interdum at interdum enim egestas. Aliquam placerat justo sed lectus lobortis ut porta nisl porttitor. Vestibulum mi dolor, lacinia molestie gravida at, tempus vitae ligula. Donec eget quam sapien, in viverra eros. Donec pellentesque justo a massa fringilla non vestibulum metus vestibulum. Vestibulum in orci quis felis tempor lacinia. Vivamus ornare ultrices facilisis. Ut hendrerit volutpat vulputate. Morbi condimentum venenatis augue, id porta ipsum vulputate in. Curabitur luctus tempus justo. Vestibulum risus lectus, adipiscing nec condimentum quis, condimentum nec nisl. Aliquam dictum sagittis velit sed iaculis. Morbi tristique augue sit amet nulla pulvinar id facilisis ligula mollis. Nam elit libero, tincidunt ut aliquam at, molestie in quam. Aenean rhoncus vehicula hendrerit.