% Chapter 6

\chapter{Derivation of PCB Layout from Schematics} % Main chapter title

\label{Chapter6} % For referencing the chapter elsewhere, use \ref{Chapter1} 

%----------------------------------------------------------------------------------------
This section provides
\section{PCB Footprint Design}

	Most of the components were either downloaded from the manufacturer's approved source (eg. Digikey) or SnapEDA, components which didn't have footprints or where the footprints had incorrect dimensions were created manually using the footprint editor in Alitum. The footprints were created by following the recommended PCB outline in the desired component's data sheet. There were a few components where the data sheet was unavailable and hence the footprints were created using precise measurements of the actual item. The following list describes the footprints which were created without a data sheet, and how they were created;

\begin{itemize}
\item The footprints for the silicon moulds for the push buttons were designed in Altium using a pre-existing official footprint used for the design of the Nintendo Gameboy. This larger footprint was modified into the separate components of the DPAD buttons, the blue push buttons and the rectangular black push buttons.
\item The footprint for the modem connector was created by measuring the dimensions of the physical component. This connection part was then added to the recommended PCB layout of the modem, which was created from the data sheet. 
\end{itemize}

\section{PCB Layout}

	The start of the PCB Design process begun with a graphical editor being used to create a mock-up of the MEGA65 phone. This mock-up included precise measurements of the height and width of the device as well as the location of all of the major components required for the device, including the FPGA pins and the 4G Modems. The mock-up was used as the template for the PCB board with all of the measurements copied into the program. \\
	Once the schematic designs had been completed the PCB board was adjusted in Alitum to accommodate four layers. This was completed by going into Layer Stack Manager in Alitum and adding in two more signal layers. All of the schematic designs were then added into the PCB file and on top of the mock-up. 

\section{Challenges}

	There were a number of challenges with designing the PCB board, mainly to do with the vast majority of components needed to be routed. Another challenge was discovered in the fact that the aim was to have a six-layered board, which in itself is highly sophisticated. It was decided early on that the majority of the tricky PCB routing and design would be completed by a professional PCB designer in Germany.\\
Another challenge was found in working out the dimensions of the PCB as well as the component placement. Due to the size requirements of the board there was not a lot of room to spare with many components sitting flush with one another and required precise measurements to fit. \\
The PCB footprint design also provided a number of challenges with some of the components either not having data sheets, or their data sheets not showing the recommended PCB layout. In these cases a physical copy of the components was obtained and precisely measured before designing the component's footprint.






%----------------------------------------------------------------------------------------





