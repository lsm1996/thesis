% Chapter Template

\chapter{Introduction} % Main chapter title

\label{Chapter1} % Change X to a consecutive number; for referencing this chapter elsewhere, use \ref{ChapterX}

%----------------------------------------------------------------------------------------
%	SECTION 1
%----------------------------------------------------------------------------------------

This introductory chapter provides an overview of the thesis goals and the basic background behind the method reasoning. The following sections present: (1) the reason for the secure smartphone project; (2) an introduction to the MEGA65 project; and (3) the scope, research questions and design requirements of the project. This chapter concludes with a brief outline of the remainder of this thesis.

\section{Computer Security has become a Shared Delusion}

The last few years has seen a disturbing upward trend in the number and severity of security vulnerabilities in modern computing devices.
Whereas traditionally these security vulnerabilities have existed primarily in software, the ever-growing complexity of computer hardware, and in particular CPU designs, now means that we are beginning to see many more hardware-based vulnerabilities.
Indeed, the complexity of the {\em errata} for a modern processor is now often longer\todo{cite a recent Intel errata document, making sure it is longer than the 6502 data sheet} than the entire specification of early generation processors \todo{cite 6502 datasheet}.
Whereas software vulnerabilities can be quickly and economically patched, this is not always the case for hardware-based vulnerabilities, as it may require replacement of the defective hardware\todo{add citation}.  It also means that when a fix is issued, that there will likely be hundreds of millions of devices that remain incurably vulnerable for the remainder of their lives \todo{cite something like this SMM https://www.theregister.co.uk/2015/08/11/memory\_hole\_roots\_intel\_processors/}.

        Modern computers are designed to achieve increasing performance and functionality through the easiest route.
        This is typically done by adding in extra layers of logic, consequently increasing sophistication and complicating the system.
        The processors of these modern computers are optimised using speculative execution, which is a technique used by high-speed processors in order to increase performance by guessing future likely execution paths \cite{RN16}.
        This complexity however can be detrimental to security, because it becomes increasingly difficult and expensive to verify the correct operation of these more complicated devices.
        This is not merely a theoretical problem: There have been a number of recent cases where vulnerabilities have been exploited, in specific the Spectre and Meltdown hardware attacks, and the Via C3 ``god mode'' bit \todo{Cite something like https://www.theregister.co.uk/2018/08/10/via\_c3\_x86\_processor\_backdoor/}.

        The recent Spectre \todo{Add citation} and Meltdown \todo{add citation} vulnerabilities are headline examples of this trend and the difficulties that such vulnerabilities create.
        Intel and the other vendors are still struggling to produce patches that do not reduce the reliability or performance of their processors, while mitigating as much of the threat posed by these vulnerabilities as possible\todo{add citation}.
        At the end of the day, the only fully effective solution is for them to develop a new generation of processors that have the defects corrected in the silicon – an extremely expensive and time-consuming process.
        Even then, it is likely that the performance of those processors will be lower, precisely because it is performance optimisations that are the direct cause of the vulnerabilities.
        Even if processors are created that correct these known issues, they too could be found to suffer from one or more hitherto unknown vulnerabilities, and themselves require replacement.
        When commerce, personal privacy and security as well as national security depends on the correct and secure operation of computers, we clearly have a problem of considerable importance before us.

        It is precisely this problem of the impossibility of feasibly verifying the security of complex modern devices that is the reason for this project, and it's reconsideration of the merits of simpler, albeit less performant architectures.
The overall project aim is to create a secure smartphone based on a purposely simple architecture, that is capable of basic telephony functions and simple productivity tasks.
Where a trade-off must be made between security and richness of functions, the trade-offs will usually tend towards security.


The system’s architecture will be based on the Commodore 64 computer, to ensure that the device not only simple enough to be secure, but to ensure that it can be sufficiently productive. In this context, the existing productivity software, tools and developer base and knowledge to be leveraged, and also exist as a kind of proof-by-example tha the resulting system will have sufficient functionality to be useful for a variety of tasks.

        
\todo{only one sentence per line in the source, please, so that source control tools are MUCH easier to use} 
  \todo{It should stay ``MEGA65'' everywhere, not ``MEGA 65'', since ``MEGA65'' is the name of the computer}

The MEGA65 project is an existing project within the Telecommunications Research Laboratory to build a Commodore 64 compatible computer, which is described in more detail in the literature review.
Its original goal was to produce and open-source reproduction of the commercially unreleased Commodore 65 (C65)\todo{cite C65 and MEGA65}.
Further, work has already commenced to create a portable version of the MEGA65, complete with touch-screen interface.
Thus it represents a system with a simple architecture and with full source code availability, which can be used within this thesis as the basis for the creation of a simple and secure communications device.

This thesis will therefore explore the adaption of the MEGA65 retro-computer to produce a simple and secure communications device, focusing
on the hardware requirements, definition and realisation.


%-----------------------------------
%	SECTION 3
%-----------------------------------

\section{Research Questions \& Scope}
The following questions were derived to help build a foundation for the research:
\begin{enumerate}
\item What power control circuitry can satisfy the security-driven requirement of the MEGA65 to be able to power down all communications hardware when required, yet remain able to be woken by that same hardware?
\item What trade-offs, if any, are required to satisfy the functional and dimensional requirements of the MEGA65 phone PCB?
\end{enumerate}

\subsection{Design Requirements}

To design a secure phone with the intentions of it being untraceable if need be, it’s underlying architecture would need to be as simplistic as possible.
To ensure that the phone can remain secure at all times the following design requirements are required:
\begin{itemize}
\item Whilst the device is in secure mode, any component which can remain stable between power cycles must be disabled, so that there is no easy mechanism for the exfiltration of data from the device.
  This includes turning off the untrusted mobile phone modem. 
\item Only need FPGA to be powered up when needed to, so the 4G modem is required to be able to wake the FPGA if the modem is on.
\item The FPGA needs to be able to turn off the phone radio, when it wants to.
\item Physical manual override power switches are required that will allow the user to power down the modem. This will effectively be a physically enforced Airplane mode feature. 
\end{itemize}
The modem also has a high-power consumption (approximately four amperes) that needs to be electronically switched.
These objectives require research into different circuit options, including power control circuitry; and state diagrams for the transitions.

%-----------------------------------
%	SECTION 3
%-----------------------------------
\section{Structure of Thesis}

Following this chapter, Chapter 2 provides background information on the project, including an indepth study on the insecurity of modern complex CPUs.\\ 
\ref{Chapter3} follows this with a list of required use cases of the secure phone with a list of functional requirements based on the needs of the use cases.\\ 
In \ref{Chapter3} a list of hardware requirements is made from the list of functional requirements.\\
First, \ref{Chapter5} begins by giving an overview of all the schematics designed from the hardware requirements.
Section 2 details the component selection and the method behind this.
Section 3 gives detail of the errors in the schematic designs and how they were solved.
Section 4 describes the testing phase of the project.\\
\ref{Chapter6} describes the derivation of the PCB layout from the schematics.
This chapter starts by giving details about the footprint design and the method behind them, with the second section going over the PCB layout and template of the phone, with the third section detailing the challenges found throughout the process.\\
\ref{Chapter7} provides the results and discussion section of the project.\\
Finally, the conclusion of the project is provided in \ref{Chapter8}.
This summarises the project's goals and results, draws conclusions, and closes the dissertation with suggestions for further research.
