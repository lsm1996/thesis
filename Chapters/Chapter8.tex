% Chapter 8

\chapter{Conclusion} % Main chapter title

\label{Chapter8} % For referencing the chapter elsewhere, use \ref{Chapter1} 

	In the preceding chapters the following research questions were answered: 
\begin{enumerate}
	\item \textit{What are the core use cases and functional requirements required to create a useful and secure communications device?}
	\item \textit{How can those functional requirements be translated into the design for a secure communications device that is feasible to be manufactured in a University research environment?}
	\item \textit{Given the necessity to include a complex cellular modem in the design, how can that device be contained, and when necessary deprived of power?}
	\item \textit{How can the complex power management requirements of this device be met?}
	\item \textit{What trade-offs, if any, are required to satisfy the functional and dimensional requirements of the MEGA65 phone PCB, within the financial and time constraints of this project?}
\end{enumerate}

	The functional and hardware requirements of the phone were found through the completion of the project, as well as the methods in how to securely deprive the modem of power. 
There were found that no trade-offs were required during the design phase, however more testing of various components could have been completed with more time as well as a physical manufactured PCB.

An evaluation of the completed work showed that the functional and hardware requirements had been defined and the power control circuitry had been completed. During this project, extensive research was required in order to find components that seamlessly integrated with existing hardware of the device. 
Some considerations when selecting a potential component included the current and voltage that the component operated at. 
In addition to component research, well evaluated plans were vital during the project. 
These plans allowed the wide range of components to be connected and interconnected clearly and concisely during the initial design stages of the project. 
Over the course of the project three major deliverables were produced, they include, the various schematics of the phone's subsystems, the PCB footprints for the components used in the phone and a template PCB design. 
	

	After completion of the project it was clear that the main goals had been accomplished, as the functional and hardware design requirements to create a secure phone had indeed been discerned. In the consequent works to this project, a high priority should be placed on improving the PCB through improved board layout, more suited component selection and PCB routing. \\
The components required to design such a device were identified and, with adequate reasoning, evaluated as to why they were chosen. 
These components then had their schematic and PCB footprint designs created in Altium. 
The following list summates the findings found throughout the project.\\

\begin{itemize}
\item The power control circuitry was created by testing a number of different components and drawing a detailed mind map before the schematic design was started.
\item Schematics were completed for every part of the phone laid out in the functional requirements.
\item A concept design was completed after the schematic design phase to aid with the PCB layout. 
\item Through using auto-route, an 8-layered PCB board was routed. 
The auto-route function failed to route the board successfully with less layers. 
\end{itemize}

%----------------------------------------------------------------------------------------




