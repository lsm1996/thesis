% Chapter 7

\chapter{Discussion} % Main chapter title

\label{Chapter7} % For referencing the chapter elsewhere, use \ref{Chapter1} 

%----------------------------------------------------------------------------------------

	After completing the project it was found that the main goals of discovering the functional and hardware requirements to design a secure phone device was accomplished. The components required to design such a device were found with adequate reasoning as to why they were chosen, with their schematic and PCB footprint designs created in Altium. The following paragraphs summate the findings found throughout the project.\\

	At the beginning of the project research questions were devised to help guide the development of the project. The following paragraphs describe the research questions and detail the results of following the research questions and whether or not they have been accomplished.\\

\begin{enumerate}
\item \textit{What are the core use cases and functional requirements required to create a useful and secure communications device?}\\

	The core use cases and functional requirements were developed in Chapter \ref{Chapter3}. 
These use cases involved the basic telephony functions of the phone, and how a user would utilise the phone to complete certain tasks. 
The functional requirements, which came about through identifying the needs in the use cases, lead to the development of the hardware requirements, which in turn lead to the creation of a concept design of a secure communications device.\\

\item \textit{How can those functional requirements be translated into the design for a secure communications device that is feasible to be manufactured in a University research environment?}\\

	The functional requirements lead to the foundation of the hardware requirements, which in turn lead to the component selection phase. 
All of the components were selected based on their feasibility to be able to soldered at the University.
Some suitable components were rejected due to the way they're soldered.\\

\item \textit{Given the necessity to include a complex cellular modem in the design, how can that device be contained, and when necessary deprived of power?}\\

	

\item \textit{How can the complex power management requirements of this device be met?}\\

	This major research question was solved through the schematic design of the power control circuitry. 
This involved researching a method for the power control to work, with various components requiring different current needs. 
This also involved searching for a suitable component, with a number of suitable components found and compared to find the optimal component. The goal to be able to design the circuit so that all communications could be powered down and powered up by the same hardware was completed, without the PCB fully completed. This also involved testing the required components to see that the powering down was occurring.\\
	
\item \textit{What trade-offs, if any, are required to satisfy the functional and dimensional requirements of the MEGA65 phone PCB, within the financial and time constraints of this project?}\\

	While designing the PCB it was found that some of the components, for example the Smart Card Connector, were quite large. 
This caused some hassle with the placements of components and the spacing around certain components. 
Due to the first revision of the phone being quite large, every desired component was able to be placed within the confines of the PCB. 
However, due to the large scale of components, some components may need to be replaced in future iterations of the device to endure that EMI and other signal interference are reduced.
In terms of the functional requirements of the MEGA65 phone, all the major components were placed in their approximate locations based on the concept drawn which can be found in Figure \ref{fig:nopcb}. The 50-pin headers, LCD connector and Touch Screen connector were all placed in their exact positions to ensure that the FPGA would fit comfortably and that the screen would align.\\

\end{enumerate}

%----------------------------------------------------------------------------------------





